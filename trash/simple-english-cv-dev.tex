\documentclass{resume} % Use the custom resume.cls style


\usepackage{float}
\restylefloat{table}

\usepackage[left=0.4 in,top=0.4in,right=0.4 in,bottom=0.4in]{geometry} % Document margins
\usepackage{graphicx}

% \usepackage{emoji}
% \usepackage{emoji}

\newcommand{\tab}[1]{\hspace{.2667\textwidth}\rlap{#1}} 
\newcommand{\itab}[1]{\hspace{0em}\rlap{#1}}
\name{Mariano Maisonnave} % Your name
% You can merge both of these into a single line, if you do not have a website.
\address{+1(902) 448-5124 \\ Halifax, NS} 
\address{\href{mailto:mariano.maisonnave@dal.ca}{mariano.maisonnave@dal.ca} \\ \href{https://cs.dal.ca/~maisonnave/google-scholar}{cs.dal.ca/$\sim$maisonnave/google-scholar} \\ \href{https://web.cs.dal.ca/~maisonnave/}{cs.dal.ca/$\sim$maisonnave}}  %

\begin{document}

\setlength{\parsep}{0 em}
\setlength{\parskip}{0.2 em}
\setlength{\itemsep}{0 em}
\setlength{\topsep}{0 em}
\setlength{\partopsep}{0 em}
%----------------------------------------------------------------------------------------
%	OBJECTIVE
%----------------------------------------------------------------------------------------
\vspace{-0.2cm}
\begin{rSection}{SUMMARY}
\vspace{-0.1cm}
\item {I am a \textbf{computer science postdoc} working at the \textbf{Faculty of Computer Science at Dalhousie University} under the supervision of Prof. Dr. Evangelos Milios. I hold a computer science Ph.D. for which I extensively studied Machine Learning (ML) and Natural Language Processing (NLP). I have vast experience programming in Python, and I'm familiar with the whole python suite for Artificial Intelligence (AI) and ML (numPy, pandas, SciPy, spaCy, Scikit-learn, HuggingFace, etc.). I have been working with ML and NLP for 7+ years. I have successfully published in scientific journals and presented my work at multiple conferences. 
\smallskip

\item Additionally, I have 6+ years of experience teaching (as either teaching assistant, head of teaching assistants or professor). I have taught courses on Object Orient Programming (for 6+ years), Computer Networking (2 years), Computer Organization (1 year), and others. 
\smallskip


 \item \textbf{Research Interests}: information extraction, event detection and term weighting in texts. Familiar with state-of-the-art ML and NLP technologies and libraries. Studied and used causality, time series, and econometric tools. Worked on building interpretable causal graphs from news articles. 
Currently working in high-recall information retrieval.
}


\end{rSection}
%----------------------------------------------------------------------------------------
%	EDUCATION SECTION
%----------------------------------------------------------------------------------------

\begin{rSection}{Education}
\smallskip
\item {\bf \large Ph.D. in Computer Science}, Universidad Nacional del Sur, Argentina \hfill {Apr 2017 --- Oct 2021}\\
\underline{GPA}: 9.86/10\\
\underline{Thesis}: Variable selection and causal discovery from  newspaper article texts. \href{https://repositoriodigital.uns.edu.ar/bitstream/handle/123456789/5827/MAISONNAVE\%20M._TESIS.pdf?sequence=2&isAllowed=y}{[link (Spanish)]}\\
\begin{table}[!ht]
\vspace{-0.7cm}
    \begin{tabular}{|p{\textwidth}}
\textit{Lead computer science researcher in a multidisciplinary project (economy and computer science) that aimed at building causal graphs from news article texts. The project involved using Natural Language Processing Tools to extract variables from news article texts and using Econometric tools to detect causality between those variables.}
\\ \underline{Publications}: J1, J2, J3, J4, C2, C3
    \end{tabular}\vspace{-0.7cm}
\end{table}


% \underline{Role:} Lead computer science researcher (development, testing, decision-making) in a multidisciplinary research project. The project involves computer science and economy researchers.  \underline{Task:}   \underline{Tools:} The project involved the development of a novel term weighting technique, the definition of a specific event detection task and the proposal of the solution for the task. Lastly, it involved the use of causal discovery tools (Granger Causality, LiNGAM, PCMCI, and others) to build a causal graph from the terms and events. 
%  \underline{Data:} The project is carried out using news articles from the New York Times news article dataset. 



% \underline{Keywords:} Machine Learning; Natural Language Processing; Econometrics and Statistics; Causality; Information Extraction; Information Retrieval.
% {\color{red} [DEGREE DESCRIPTION]}
% Relevant courses: Econometrics, 
\bigskip

\item {\bf \large Computer Systems Engineer}, Universidad Nacional del Sur, Argentina \hfill {Mar 2011 --- Oct 2016}\\
\underline{GPA}: 8.58/10\\
\underline{Final project}: Feature engineering techniques evaluated on supervised learning methods.\\
% \underline{Keywords}: Software Development; Object Oriented Programming; Security and Computer Networking; Artificial Intelligence; Computation Theory; Computer Organization and Architectures; Databases; Embedded Systems and Electronics; Distributed Systems; Calculus. 
%Minor in Linguistics \smallskip \\
%Member of Eta Kappa Nu \\
%Member of Upsilon Pi Epsilon \\

\end{rSection}
\vspace{-0.2cm}
%----------------------------------------------------------------------------------------
% TECHINICAL STRENGTHS	
%----------------------------------------------------------------------------------------
\begin{rSection}{SELECTED SKILLS (NOT EXHAUSTIVE)}
\vspace{-0.2cm}
\begin{table}[!ht]
\resizebox{\textwidth}{!}{
\begin{tabular}{llllll}
\textbf{Languages}        & Python                  & Java  & C             & C\#                      & SQL                  \\
\textbf{Sysadmin Skills}  & Bash and Bash Scripting & VIM   & Git           & Docker                   & SSH/SFTP             \\
\textbf{Python Libraries} & Pandas                  & NumPy & Matplotlib    & Jupyter Notebook         & Seaborn              \\
\textbf{ML Libraries}     & Scikit-learn            & Keras & HuggingFace   & SciPy                    & SentenceTransformers \\
\textbf{NLP Libraries}    & NLTK                    & spaCy & BeautifulSoup & re (Regular Expressions) & Gensim              
\end{tabular}
}
\end{table}\vspace{-0.2cm}
% Python; NumPy; Pandas; Matplotlib; Jupyter Notebook; Scikit-learn; spaCy; NLTK; Keras+TensorFlow; GNU/Linux and Bash; Git; Docker; SQL and VIM. 
% Git, Linux and Bash, Keras with TensorFlow, Scikit-learn, Hugging Face, NLTK, , Jupyter, Pandas, Docker, SQL, Numpy, spaCy, and vim.
% \begin{tabular}{ @{} >{\bfseries}l @{\hspace{6ex}} l }
% Selected Technical Skills & Python, Git, Linux and Bash, Keras, Scikit-learn, 
% % \\
% % Soft Skills & A, B, C, D\\
% % XYZ & A, B, C, D\\
% \end{tabular}\\
\end{rSection}


\begin{rSection}{WORK EXPERIENCE}
\item \textbf{\large Postdoctoral Employee} \hfill Nov 2021 --- Present\\\smallskip
Faculty of Computer Science, Dalhousie University \hfill \textit{Halifax, Canada}\\
\begin{table}[!ht]
\vspace{-0.7cm}
    \begin{tabular}{|p{\textwidth}}
\underline{Project title}: \textbf{Visual analytics for text-intensive social science research on immigration}\\
\textit{Lead computer science researcher in multidisciplinary (computer science and social science) applied research project that aims to enable novel text-intensive research in social science through the development of novel computer science tools. The project currently involves high-recall information retrieval on a large corpus of news articles and the organization of the results through topic modelling. }
    \end{tabular}\vspace{-0.7cm}
\end{table}

% \textit{  \underline{Role:} Development, testing and decision-making in the DETESTS shared task, which is part of IberLEF 2022. 
%  \underline{Task:} Detect and classify stereotypes in sentences from comments posted in Spanish in response to different online news articles related to immigration.
%  \underline{Tools:} Our approach consists of a Multi-Task Learning strategy applied to pre-trained deep language models, which allows learning a sequence representation for each comment.
%  \underline{Data:} The corpora consist of comments published in response to different articles extracted from Spanish online newspapers and discussion forums. The corpus is provided by the shared task organizers. }


% \textit{ \underline{Role:} Lead computer science researcher (development, testing, decision-making) in a multidisciplinary applied research project. The project involves computer science and social science researchers. 
%  \underline{Task:} Support text-intensive research in social science through the development of novel computer science tools. 
%  \underline{Tools:} The project involves the development and testing of a high-recall information retrieval system. The problem is solved using an active learning approach followed by topic modelling.
%  \underline{Data:} The project is carried out using news articles from the Toronto Star and Globe and Mail newspapers.}

\newpage

\item  \textbf{Servicios Tecnológicos de Alto Nivel (STAN)} (\textit{Highly skilled transfer services})\hfill May 2022 --- Present\\\smallskip
Global News Group \& Universidad Nacional del Sur \hfill \textit{Bahía Blanca, Argentina}\\
\begin{table}[!ht]
\vspace{-0.7cm}
    \begin{tabular}{|p{\textwidth}}
    \underline{Project title}: \textbf{Real-time event and story detection from news articles} \\
\textit{Co-directing two graduated students in an applied research project. The project aims at analyzing news article texts and grouping them into cohesives stories which we further organize into chapters.}
    \end{tabular}\vspace{-0.5cm}
\end{table}


% \textit{\underline{Project title}: \textit{Real-time event and story detection from news articles   }
%  \underline{Role:} 
%  \underline{Task:} Analyzing news article texts and grouping them into cohesives stories which we further organize into chapters.
%  \underline{Tools:} The project involves time-aware clustering, classification, and event detection using state-of-the-art deep language models for representation, training (fine-tuning) and predicting.
%  \underline{Data:} The project is carried out using news articles provided by the media monitoring company Global News Group.}
 

\item  \textbf{Servicios Tecnológicos de Alto Nivel (STAN)} (\textit{Highly skilled transfer services})\hfill Oct 2020 --- Present\\\smallskip
BID-INTAL \& Universidad Nacional del Sur \hfill \textit{Bahía Blanca, Argentina}\\
\begin{table}[!ht]
\vspace{-0.7cm}
    \begin{tabular}{|p{\textwidth}}
\underline{Project title}: \textbf{Foreign direct investment monitoring, detection and collection from Twitter}\\
\textit{Co-directing a graduated student in an applied research project. The project aims at Detecting Foreign Direct Investment (FDI) mentions in tweets, followed by the extraction of structured information about the detected FDI. }
    \end{tabular}\vspace{-0.5cm}
\end{table}


% \textit{ \underline{Role:} Co-directing a graduated student in an applied research project.
%  \underline{Task:} Detecting Foreign Direct Investment (FDI) mentions in tweets, followed by the extraction of structured information about the FDI (origin country and organization, target country and organization, money amount). 
%  \underline{Tools:} The project involves text classification and information extraction using state-of-the-art deep language models. 
%  \underline{Data:} The project is carried out using Tweets collected specifically for this project using the Twitter API. }

\end{rSection}

\begin{rSection}{TEACHING WORK EXPERIENCE}

\textbf{Professor} \hfill  2020\\
Universidad Nacional del Sur \hfill \textit{Bahía Blanca, Argentina}
\\\textit{Course:} Analysis and Problem Solving


\bigskip
\textbf{Head of Teaching Assistants} \hfill  2019 --- Present\\
Universidad Nacional del Sur \hfill \textit{Bahía Blanca, Argentina}
\\\textit{Course:} Object Oriented Programming


\bigskip
\textbf{Teaching Assistant} \hfill  2015 --- Present\\
Universidad Nacional del Sur \hfill \textit{Bahía Blanca, Argentina}
\\ Courses
\begin{itemize}
    \item Object Oriented Programming\vspace{-0.1cm}
    \item Embedded Systems \vspace{-0.1cm}
    \item Computer Networks   \vspace{-0.1cm}
    \item Computer Organization  \vspace{-0.1cm}
    \item Analysis and Problem Solving  \vspace{-0.1cm}
    \item Formal Languages and Automata Theory
\end{itemize}

% \textbf{Head of Teaching Assistants --- Object Oriented Programing Course} \hfill  2019 - Present\\
% Universidad Nacional del Sur \hfill \textit{Bahía Blanca, Argentina}
% \vspace{-0.1cm}

% \textbf{Teaching Assistant --- Embedded Systems \& Computer Networks} \hfill  2017 - Present\\
% Universidad Nacional del Sur \hfill \textit{Bahía Blanca, Argentina}
% \vspace{-0.1cm}

% \textbf{Teaching Assistant --- Computer Organization} \hfill  2020\\
% Universidad Nacional del Sur \hfill \textit{Bahía Blanca, Argentina}
% \vspace{-0.1cm}

% \textbf{Professor --- Analysis and Problem Solving} \hfill  2020\\
% Universidad Nacional del Sur \hfill \textit{Bahía Blanca, Argentina}
% \vspace{-0.1cm}

% \textbf{Teaching Assistant --- Object Oriented Programing Course} \hfill  2015 -  2019\\
% Universidad Nacional del Sur \hfill \textit{Bahía Blanca, Argentina}
% \vspace{-0.1cm}

% \textbf{Teaching Assistant --- Analysis and Problem Solving} \hfill  2017 - 2018\\
% Universidad Nacional del Sur \hfill \textit{Bahía Blanca, Argentina}
% \vspace{-0.1cm}

% \textbf{Teaching Assistant --- Formal Languages and Automata Theory} \hfill  2015 - 2016\\
% Universidad Nacional del Sur \hfill \textit{Bahía Blanca, Argentina}


\end{rSection}


\begin{rSection}{SELECTED PUBLICATIONS}
\item \textbf{Maisonnave, M.}, Delbianco, F., Tohmé, F., Milios, E. and Maguitman, A.G., \textbf{2022}. \textit{Causal graph extraction from news: a comparative study of time-series causality learning techniques}. PeerJ Computer Science, 8, p.e1066. 
\href{https://peerj.com/articles/cs-1066/}{[link]}

\item \textbf{Maisonnave, M.}, Delbianco, F., Tohmé, F., Maguitman, A. and Milios, E., \textbf{2022}. \textit{Detecting ongoing events using contextual word and sentence embeddings}. Expert Systems with Applications, 209, p.118257. 
\href{https://www.sciencedirect.com/science/article/pii/S0957417422013975}{[link]}

\item \textbf{Maisonnave, M.}, Delbianco, F., Tohmé, F. and Maguitman, A., \textbf{2021}. \textit{Assessing the behavior and performance of a supervised term-weighting technique for topic-based retrieval}. Information Processing \& Management, 58(3), p.102483. 
\href{https://www.sciencedirect.com/science/article/pii/S0306457320309729}{[link]}

\end{rSection} 

\vspace{-0.2cm}
% \begin{rSection}{CONGRESS PUBLICATIONS}
% \item[C1]  Ramirez-Orta, J., Sabando, M.V.,\textbf{ Maisonnave, M.} and Milios, E., \textbf{2022}. \textit{MALNIS at IberLEF-2022 DETESTS Task: A Multi-Task Learning Approach for Low-Resource Detection of Racial Stereotypes in Spanish}. In Proceedings of the Iberian Languages Evaluation Forum (IberLEF 2022). CEUR Workshop Proceedings, CEUR-WS. org.
% \href{https://ceur-ws.org/Vol-3202/detests-paper2.pdf}{[link]}

% \item[C2]  \textbf{Maisonnave, M.}, Delbianco, F., Tohmé, F., Maguitman, A.G. and Milios, E.E., \textbf{2020}, September, \textit{Assessing Causality Structures learned from Digital Text Media}, In Proceedings of the ACM Symposium on Document Engineering 2020 (pp. 1-4)
% \href{https://dl.acm.org/doi/10.1145/3395027.3419594}{[link]}

% \item[C3]  \textbf{Maisonnave, M.}, Delbianco, F., Tohmé, F.A. and Maguitman, A.G., \textbf{2018}, November, \textit{A Supervised Term-Weighting Method and its Application to Variable Extraction from Digital Media}, In XIX Simposio Argentino de Inteligencia Artificial (ASAI)-JAIIO 47.
% \href{https://47jaiio.sadio.org.ar/sites/default/files/ASAI-07.pdf}{[link]}

% \end{rSection}


% \begin{rSection}{CONGRESS PRESENTATIONS}
% \item  \textbf{Maisonnave, M.}, \textbf{2022}, November, \textit{Computer-Assisted Text-Intensive Social Science Research on Immigration}, In the 26$^{th}$ Biennial Canadian Ethnic Studies Association Conference.

% \item  \textbf{Maisonnave, M.}, \textbf{2019}, June, \textit{Detección de Textos Similares a través de una Técnica de Agrupamiento Basada en Densidad} (\textit{Text similarity dectection through a Density-Based Clustering Technique}), Comunicación en XV Congreso Dr. Antonio Monteiro.

% \end{rSection}


% \begin{rSection}{PARTICIPATION IN SHARED TASKS}
% \vspace{-0.1cm}
% \item \textbf{A multi-task learning approach for low-resource detection of racial stereotypes in Spanish} \hfill 2022\\
% Dalhousie University \hfill \textit{Halifax, Canada}\\
% \begin{table}[!ht]
% \vspace{-0.7cm}
%     \begin{tabular}{|p{\textwidth}}
% \textit{Participation in the DETESTS shared task, which was part of IberLEF 2022. The shared tasks involved detecting and classifying stereotypes in sentences from comments posted in Spanish in response to different online news articles related to immigration.}\\
% \underline{Publications}: C1
%     \end{tabular}\vspace{-0.3cm}
% \end{table}


% \textit{  \underline{Role:} Development, testing and decision-making in the DETESTS shared task, which is part of IberLEF 2022. 
%  \underline{Task:} Detect and classify stereotypes in sentences from comments posted in Spanish in response to different online news articles related to immigration.
%  \underline{Tools:} Our approach consists of a Multi-Task Learning strategy applied to pre-trained deep language models, which allows learning a sequence representation for each comment.
%  \underline{Data:} The corpora consist of comments published in response to different articles extracted from Spanish online newspapers and discussion forums. The corpus is provided by the shared task organizers. }

% \end{rSection}
% \newpage


% \begin{rSection}{CONTRIBUTED OPEN-SOURCE DATASETS}
% \textbf{Maisonnave, M.}, Delbianco, F., Tohmé, F., Maguitman, A., Milios, E., \textbf{2020}, \textit{Event Detection Dataset}, Mendeley Data, V1, doi: \href{https://dx.doi.org/10.17632/7d54rvzxkr.1}{10.17632/7d54rvzxkr.1}

% \textbf{Maisonnave, M.}, Delbianco, F., Tohmé, Fernando; Maguitman, A., \textbf{2019}, \textit{Economic Relevant News from The Guardian}, Mendeley Data, V3, doi:  \href{https://dx.doi.org/10.17632/yt8j2f3hpp.3}{10.17632/yt8j2f3hpp.3}
% \end{rSection}


\begin{rSection}{AWARDS AND SCHOLARSHIPS} 
\begin{itemize}
    \item Three-year winner of the Google Latin America Research Award (Google LARA) (2019, 2020 and 2021). Project: ``\textit{Learning Causal Models from Digital Media}.''
    \item Recipient of the Emerging Leaders in the Americas Program (ELAP) scholarship, 2018. Funding for a research stay at Dalhousie University from December 2018 to April 2019.
    \item Recipient of a five-year scholarship to pursue a doctorate in computer science, 2017-2022. Granted by the Argentinian National Scientific and Technical Research Council (CONICET).
    \item Recipient of the Encouragement of Scientific Vocations Scholarship (EVC-CIN), 2015 and 2016. Funding for undergrad students to conduct a part-time research project.
\end{itemize}

\end{rSection}


% \begin{rSection}{SELECTED PROJECTS (NOT EXHAUSTIVE)}


% %%%%%%%%%%%%%%%%%%%
% % # GLOBAL NEWS # %
% %%%%%%%%%%%%%%%%%%%
% %   \textbf{Real-time event and story detection from news articles } \hfill May 2022 - Present\\\smallskip
% % Global News Group \& Universidad Nacional del Sur \hfill \textit{Bahía Blanca, Argentina}\\


% %%%%%%%%%%%%%%%
% % # POSTDOC # %
% %%%%%%%%%%%%%%%
% %  \textbf{Visual analytics for text-intensive social science research on immigration} \hfill Nov 2021 - Present\\
% % Dalhousie University \hfill \textit{Halifax, Canada}\\

%  \href{{https://github.com/mmaisonnave/high-recall-information-retrieval-system}}{[github]}

 
% %    \textbf{Foreign direct investment monitoring, detection and collection from Twitter} \hfill Oct 2020 - Present\\
% % BID-INTAL  \& Universidad Nacional del Sur \hfill \textit{Bahía Blanca, Argentina}\\
 

%    % \textbf{A multi-task learning approach for low-resource detection of racial stereotypes in Spanish} \hfill 2022\\
%  % Dalhousie University \hfill \textit{Halifax, Canada}\\


  
%  %   \textbf{Variable selection and causal discovery from news article texts} \hfill Apr 2017 - Oct 2021\\
%  % Universidad Nacional del Sur \hfill \textit{Bahía Blanca, Argentina}\\
  

% \end{rSection} 






%----------------------------------------------------------------------------------------
%	WORK EXPERIENCE SECTION
%----------------------------------------------------------------------------------------

% \begin{rSection}{PROJECTS}
% \vspace{-1.25em}
% \item \textbf{Hiring Search Tool.} {Built a tool to search for Hiring Managers and Recruiters by using ReactJS, NodeJS, Firebase and boolean queries. Over 25000 people have used it so far, with 5000+ queries being saved and shared, and search results even better than LinkedIn! \href{https://hiring-search.careerflow.ai/}{(Try it here)}}
% \item \textbf{Short Project Title.} {Build a project that does something and had quantified success using A, B, and C. This project's description spans two lines and also won an award.}
% \item \textbf{Short Project Title.} {Build a project that does something and had quantified success using A, B, and C. This project's description spans two lines and also won an award.}
% \end{rSection} 

%----------------------------------------------------------------------------------------
% \begin{rSection}{Extra-Curricular Activities} 
% \begin{itemize}
%     \item 	Actively write \href{https://www.faangpath.com/blog/}{blog posts} and social media posts (\href{https://www.tiktok.com/@faangpath}{TikTok}, \href{https://www.instagram.com/faangpath/?hl=en}{Instagram}) viewed by over 20K+ job seekers per week to help people with best practices to land their dream jobs. 
%     \item	Sample bullet point.
% \end{itemize}


% \end{rSection}

%----------------------------------------------------------------------------------------

\newpage
\begin{rSection}{UNIVERSITY SERVICE}

\item \textbf{Mentoring Program --- Coordinator} \hfill  2021\\
Universidad Nacional del Sur \hfill \textit{Bahía Blanca, Argentina}

\item \textbf{Mentoring Program --- Mentor} \hfill  2015 --- 2020\\
Universidad Nacional del Sur \hfill \textit{Bahía Blanca, Argentina}

\item \textbf{Computer Science Student Union --- Treasurer} \hfill  2016\\
Universidad Nacional del Sur \hfill \textit{Bahía Blanca, Argentina}

\item \textbf{Computer Science Departmental Council --- Student Elected Member} \hfill  2015\\
Universidad Nacional del Sur \hfill \textit{Bahía Blanca, Argentina}

\item \textbf{Computer Science Student Union --- Vice President} \hfill  2015\\
Universidad Nacional del Sur \hfill \textit{Bahía Blanca, Argentina}

\item \textbf{Computer Science Academic Affairs Commission --- Student Member} \hfill  2015\\
Universidad Nacional del Sur \hfill \textit{Bahía Blanca, Argentina}

\end{rSection}


\begin{rSection}{CONTACT REFERENCES}
\vspace{-0.2cm}

\setlength{\itemsep}{0.32em}

% \setlength{\parsep}{0 em}
\setlength{\parskip}{0.2 em}
\setlength{\itemsep}{0 em}
\setlength{\topsep}{0 em}
\setlength{\partopsep}{0 em}
\item \textbf{Ana G. Maguitman}\\
Research Professor, \textit{Universidad Nacional del Sur} \hfill \texttt{agm@cs.uns.edu.ar} \\
\href{https://scholar.google.com.ar/citations?user=upxByNEAAAAJ&hl=en&oi=ao}{Google Scholar}

\item \textbf{Evangelos E. Milios}\\
University Research Professor \& Deep Sense Scientific Director, \textit{Dalhousie University} \hfill \texttt{eem@cs.dal.ca}\\
\href{https://scholar.google.com.ar/citations?user=ME8aQywAAAAJ&hl=en&oi=ao}{Google Scholar}

\item \textbf{Axel J. Soto}\\
Researcher, \textit{Universidad Nacional del Sur} \hfill \texttt{axel.soto@cs.uns.edu.ar}\\
\href{https://scholar.google.com.ar/citations?user=AlwbL8IAAAAJ&hl=en&oi=ao}{Google Scholar}
\end{rSection}
% \begin{rSection}{UNIVERSITY SERVICE}
% \end{rSection}

% %----------------------------------------------------------------------------------------
% \begin{rSection}{Leadership} 
% \begin{itemize}
%     \item Admin for the \href{https://discord.com/invite/WWbjEaZ}{FAANGPath Discord community} with over 6000+ job seekers and industry mentors. Actively involved in facilitating online events, career conversations, and more alongside other admins and a team of volunteer moderators! 
% \end{itemize}


% \end{rSection}


\end{document}
