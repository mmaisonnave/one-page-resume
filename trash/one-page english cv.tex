\documentclass{resume} % Use the custom resume.cls style

\usepackage[left=0.4 in,top=0.4in,right=0.4 in,bottom=0.4in]{geometry} % Document margins

\usepackage{xcolor}
% \usepackage{emoji}
% \usepackage{emoji}

\newcommand{\tab}[1]{\hspace{.2667\textwidth}\rlap{#1}} 
\newcommand{\itab}[1]{\hspace{0em}\rlap{#1}}
\name{Mariano Maisonnave, Ph.D.} % Your name
% You can merge both of these into a single line, if you do not have a website.
\address{+1(902) 448-5124 \\ Halifax, NS}
\address{\href{mailto:mariano_maisonnave@cbu.ca}{mariano\_maisonnave@cbu.ca} \\  
\href{https://cs.dal.ca/~maisonnave/google-scholar}{Google Scholar}} 
% \\ \href{https://mmaisonnave.github.io/}{mmaisonnave.github.io}
% }     %
% \address{\color{gray}(he/him)}

\begin{document}

%----------------------------------------------------------------------------------------
%	OBJECTIVE
%----------------------------------------------------------------------------------------


\vspace{-0.4cm}
\begin{rSection}{}
I am a \textbf{computer science postdoc} with 8+ years of Machine Learning (ML) experience. My research projects involved applying state-of-the-art natural language processing, causality, time series analysis, and explainable ML tools in economics, energy, healthcare, and social science domains. I have several published articles in high-impact journals and conferences.
\end{rSection}



\vspace{-0.1cm}
\begin{rSection}{SELECTED SKILLS}
\textbf{Programming:} Proficient with \texttt{Python}, \texttt{Java}, \texttt{C}, and \texttt{Git}.
\vspace{-0.2cm}

\textbf{Python Libraries:} Highly skilled with all major \texttt{Python} libraries (such as \texttt{NumPy}, \texttt{pandas}, \texttt{Matplotlib}).
\vspace{-0.2cm}

\textbf{Machine Learning:} Extensive experience with ML libraries (\texttt{spaCy}, \texttt{PyTorch}, \texttt{scikit-learn}, \texttt{Gensim}, and others).
\vspace{-0.7cm}

\textbf{Sysadmin:} Skilled \texttt{Linux} enthusiast experienced in remote computing resources (\texttt{SSH/SFTP}, \texttt{Bash}, \texttt{Slurm}).
\end{rSection}



\vspace{-0.1cm}
\begin{rSection}{EXPERIENCE}
\textbf{Postdoc Fellow  } \hfill Aug 2023 --- Present\\Cape Breton University \hfill \textit{Sydney, NS}\\
\vspace{-0.7cm}

\begin{itemize}
    \item \textbf{Overview}: Research focused on explainable ML applied to the real-world data in the healthcare domain.
    \vspace{-0.25cm}
    \item \textbf{Goal}: identify risk factors and obtain insight for medical personnel to handle the unplanned hospital readmission problem in Nova Scotia.
    \vspace{-0.25cm} 
    \item \textbf{Responsabilities}: handle the entire lifecycle of our research, formulating research questions, conducting literature review, executing experiments, writing scientific articles and presenting results in relevant venues.
\end{itemize}
% Our research focused on using explainable ML applied to the real-world data in the healthcare domain to identify risk factors and obtain insight for medical personnel to handle the unplanned hospital readmission problem in Nova Scotia. I handle the entire lifecycle of our research. From formulating research questions, conducting thorough literature reviews, executing experiments to produce both theoretically grounded and applicable solutions, and writing scientific articles and presenting our results in relevant venues.

% \begin{itemize}
%     \item Research focused on explainable ML applied to real-world healthcare data from Nova Scotia, Canada.
%     \vspace{-0.35cm}
%     \hspace{-1cm} \item  Aim: Identify risk factors and insights for medical personnel to handle unplanned hospital readmissions.
%     \vspace{-0.35cm}
%     \item Responsibilities:
%     \vspace{-0.35cm}
%     \begin{itemize}
%             \item Formulating research questions.
%             \vspace{-0.35cm}
%             \item Conducting thorough literature reviews.
%             \vspace{-0.35cm}
%             \item Executing experiments to produce theoretically grounded and applicable solutions.
%             \vspace{-0.35cm}
%             \item Writing scientific articles and presenting results in relevant venues.
%     \end{itemize}
% \end{itemize}


% \begin{itemize}
%     \item \textit{\textcolor{darkgray}{Mariano Maisonnave, Enayat Rajabi, Majid, and Peter, 2024. Alternate Level of Care Patients in Canada: A Systematic Review: Alternate Level of Care Patients in Canada. Accepted for publication in Canadian Geriatrics Journal.}}
% \end{itemize}


\textbf{Postdoc Employee} \hfill Nov 2021 --- Jul 2023\\
Dalhousie University \hfill \textit{Halifax, NS}\\
\vspace{-0.7cm}
\begin{itemize}
    \item \textbf{Goal}: to review the state-of-the-art in active learning for high-recall information and to provide solution to social science researchers conducting their research using large amounts of textual data.
    \vspace{-0.25cm}
    \item \textbf{Responsabilities}: designed and implemented an active learning system in Python that enabled researchers to interact with a ML model and train it in real-time.
    \vspace{-0.25cm}
    \item \textbf{Achievements}: Social science researchers could perform information retrieval over millions of news articles using our system, which led them to conduct novel research in their fields.
\end{itemize}

% I reviewed the state-of-the-art active learning for high-recall information retrieval to provide a solution to social science researchers conducting their research using large amounts of textual data. I designed and implemented an active learning system in Python that enabled researchers to interact with a ML model and train it in real-time. Social science researchers could perform information retrieval over millions of news articles using our system, which led them to conduct novel research in their fields.
% \begin{itemize}
%     \item Reviewed state-of-the-art active learning for high-recall information retrieval.
    
%     \item Aim: Provide a solution for social science researchers using large amounts of textual data.
%     \item Responsibilities:
%     \item Designed and implemented an active learning system in Python.
%     \item Enabled researchers to interact with a ML model and train it in real-time.
%     \item System allowed information retrieval over millions of news articles.
%     \item Facilitated novel research in social science fields.
% \end{itemize}


\textbf{Research Fellow} \hfill Nov 2021 --- Jul 2023\\
Institute for Computer Science and Engineering \hfill \textit{Bahía Blanca, Argentina}\\
\vspace{-0.7cm}
\begin{itemize}
    \item \textbf{Overview}: Interdisciplinary project that aims to extract causal graphs from a large corpus of news article texts using causality learning and time series analysis tools (primarily in Python, some in R), natural language processing, and ML tools.
    \vspace{-0.25cm}
    \item \textbf{Achievements: }
    \vspace{-0.25cm}
    \begin{itemize}
    \item  I developed a term-weighting technique to identify relevant concepts present in texts.
    \vspace{-0.25cm}
    \item  I built and evaluated a ML model to identify ongoing events from news articles.
    \vspace{-0.25cm}
    \item We used causal learning tools on top of the extracted variables (concepts and events) to condense the information from the text into a single causal graph.
    \vspace{-0.25cm}
    \item Applied causal learning tools to a real-world energy demand time-series dataset to understand the role of causality in forecasting analysis.
    \vspace{-0.25cm}
    \item \textbf{Achievements}: We published articles on term weighting, causal learning, natural language processing, and energy forecasting in several high-impact journals and conferences.
    \end{itemize}
\end{itemize}

% I collaborated closely with Computer Science and Economics researchers to design a framework to extract causal graphs from a large corpus of news article texts. I developed a term-weighting technique to identify relevant concepts present in texts. I built and evaluated a ML model to identify ongoing events from news articles. We used causal learning tools on top of the extracted variables (concepts and events) to condense the information from the text into a single causal graph. As a case study, we also applied causal learning tools to a real-world energy demand time-series dataset to understand the role of causality in forecasting analysis. 

% I used causality learning and time series analysis tools (primarily in Python, some in R), natural language processing, and ML tools to conduct the project. 
% We published articles on term weighting, causal learning, natural language processing, and energy forecasting in several high-impact journals and conferences.





\textbf{Teaching} \hfill Oct 2015 --- July 2024\\
Universidad Nacional del Sur \hfill \textit{Bahía Blanca, Argentina}\\
\vspace{-0.7cm}

\begin{itemize}
    \item Teaching assistant, head of teaching assistant, and professor at several university undergraduate courses.
    \vspace{-0.25cm}
    \item courses for which I taught the longest were Object Orient Programming in \texttt{Java}, Computer Networking in \texttt{C} and \texttt{Python}, and Embbed Systems in \texttt{C}. 
\end{itemize}
%  \begin{itemize}
%     \itemsep -3pt {} 
%      \item Work as teaching assistant, head of teaching assistant and professor of several undergrad courses.
%      \item Courses: computer networking, embedded systems and introduction to object-oriented programming. 

%  \end{itemize}
\end{rSection} 
%----------------------------------------------------------------------------------------
%	EDUCATION SECTION
%----------------------------------------------------------------------------------------




\vspace{-0.1cm}
\begin{rSection}{Education}

{\bf Ph.D. in Computer Science} (GPA 9.86/10), Universidad Nacional del Sur, Argentina \hfill {2017 --- 2021}\\
\underline{Thesis}: Variable selection and causal discovery from  newspaper article texts. \href{https://repositoriodigital.uns.edu.ar/bitstream/handle/123456789/5827/MAISONNAVE\%20M._TESIS.pdf?sequence=2&isAllowed=y}{[link (Spanish)]}\\
% Relevant courses: Econometrics, 

\vspace{-0.6cm}


{\bf Computer Systems Engineer} (GPA 8.58/10), Universidad Nacional del Sur, Argentina \hfill {2011 --- 2016}\\
\underline{Final project}: Feature engineering techniques evaluated on supervised learning methods.
%Minor in Linguistics \smallskip \\
%Member of Eta Kappa Nu \\
%Member of Upsilon Pi Epsilon \\


\end{rSection}

%----------------------------------------------------------------------------------------
% TECHINICAL STRENGTHS	
%----------------------------------------------------------------------------------------




% \begin{rSection}{SELECTED PUBLICATIONS}
% \vspace{-1.25em}
% \item \textbf{Maisonnave, M}., Delbianco, F., Tohmé, F., Milios, E. and Maguitman, A.G., 2022. \textit{Causal graph extraction from news: a comparative study of time-series causality learning techniques}. PeerJ Computer Science, 8, p.e1066. \href{https://peerj.com/articles/cs-1066/}{[link]}
% \vspace{-0.2cm}
% \item \textbf{Maisonnave, M}., Delbianco, F., Tohmé, F., Maguitman, A. and Milios, E., 2022. \textit{Detecting ongoing events using contextual word and sentence embeddings}. Expert Systems with Applications, 209, p.118257. \href{https://www.sciencedirect.com/science/article/pii/S0957417422013975}{[link]}
% \vspace{-0.2cm}
% \item \textbf{Maisonnave, M}., Delbianco, F., Tohmé, F. and Maguitman, A., 2021. \textit{Assessing the behavior and performance of a supervised term-weighting technique for topic-based retrieval}. Information Processing \& Management, 58(3), p.102483. \href{https://www.sciencedirect.com/science/article/pii/S0306457320309729}{[link]}
% % \vspace{-0.15cm}
% \vspace{-0.2cm}
% \end{rSection} 
%----------------------------------------------------------------------------------------
%	WORK EXPERIENCE SECTION
%----------------------------------------------------------------------------------------

% \begin{rSection}{PROJECTS}
% \vspace{-1.25em}
% \item \textbf{Hiring Search Tool.} {Built a tool to search for Hiring Managers and Recruiters by using ReactJS, NodeJS, Firebase and boolean queries. Over 25000 people have used it so far, with 5000+ queries being saved and shared, and search results even better than LinkedIn! \href{https://hiring-search.careerflow.ai/}{(Try it here)}}
% \item \textbf{Short Project Title.} {Build a project that does something and had quantified success using A, B, and C. This project's description spans two lines and also won an award.}
% \item \textbf{Short Project Title.} {Build a project that does something and had quantified success using A, B, and C. This project's description spans two lines and also won an award.}
% \end{rSection} 

%----------------------------------------------------------------------------------------
% \begin{rSection}{Extra-Curricular Activities} 
% \begin{itemize}
%     \item 	Actively write \href{https://www.faangpath.com/blog/}{blog posts} and social media posts (\href{https://www.tiktok.com/@faangpath}{TikTok}, \href{https://www.instagram.com/faangpath/?hl=en}{Instagram}) viewed by over 20K+ job seekers per week to help people with best practices to land their dream jobs. 
%     \item	Sample bullet point.
% \end{itemize}


% \end{rSection}

%----------------------------------------------------------------------------------------
\begin{rSection}{AWARDS AND SCHOLARSHIPS} 
\begin{itemize}
    \item Three-year winner of the Google Latin America Research Award (Google LARA) (2019, 2020 and 2021). Project: ``\textit{Learning Causal Models from Digital Media}.''
    \vspace{-0.2cm}
    \item Recipient of the Emerging Leaders in the Americas Program (ELAP) scholarship, 2018. Funding for a research stay at Dalhousie University from December 2018 to April 2019.
    % \item Recipient of a five-year scholarship to pursue a doctorate in computer science, 2017-2022. Granted by the Argentinian National Scientific and Technical Research Council (CONICET).
    % \vspace{-0.2cm}     
    % \item Recipient of the Encouragement of Scientific Vocations Scholarship (EVC-CIN), 2015 and 2016. Funding for undergrad students to conduct a part-time research project.
\end{itemize}

\end{rSection}
% %----------------------------------------------------------------------------------------
% \begin{rSection}{Leadership} 
% \begin{itemize}
%     \item Admin for the \href{https://discord.com/invite/WWbjEaZ}{FAANGPath Discord community} with over 6000+ job seekers and industry mentors. Actively involved in facilitating online events, career conversations, and more alongside other admins and a team of volunteer moderators! 
% \end{itemize}


% \end{rSection}


\end{document}
