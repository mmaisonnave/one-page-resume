\documentclass{resume} % Use the custom resume.cls style

\usepackage{marvosym}
\usepackage[left=0.4 in,top=0.4in,right=0.4 in,bottom=0.4in]{geometry} % Document margins

\usepackage{xcolor}
% \usepackage{emoji}
% \usepackage{emoji}

\newcommand{\tab}[1]{\hspace{.2667\textwidth}\rlap{#1}} 
\newcommand{\itab}[1]{\hspace{0em}\rlap{#1}}
\name{Mariano Maisonnave} % Your name
% You can merge both of these into a single line, if you do not have a website.
\address{+1(902) 448-5124 \\ Lower Sackville, NS}
\address{
\href{https://www.linkedin.com/in/mariano-maisonnave/}{LinkedIn/in/mariano-maisonnave/} \\
\href{mailto:maisonnavemariano@gmail.com}{maisonnavemariano@gmail.com} \\ 
\href{https://cs.dal.ca/~maisonnave/google-scholar}{Google Scholar} 
} 


% \\ \href{https://mmaisonnave.github.io/}{mmaisonnave.github.io}
% }     %
% \address{\color{gray}(he/him)}

\begin{document}

%----------------------------------------------------------------------------------------
%	OBJECTIVE
%----------------------------------------------------------------------------------------
\vspace{-0.3cm}
\begin{rSection}{}
Experienced \textbf{postdoctoral researcher} with 8+ years in Machine Learning based in Halifax. Expertise in Large Language Models (LLMs), NLP, causality analysis, time series analysis, and explainable ML tools. Proven track record of impactful research in computer science, energy, healthcare, and social sciences.
\end{rSection}





\vspace{-0.2cm}
\begin{rSection}{SELECTED SKILLS}
% \vspace{-0.2cm}
\textbf{Programming Languages and tools:} Proficient with \texttt{Python}, \texttt{Java}, \texttt{C}, and \texttt{Git}.
\vspace{-0.25cm}

\textbf{Python Libraries:} Highly skilled with all major \texttt{Python} libraries (such as \texttt{NumPy}, \texttt{pandas}, \texttt{Matplotlib}).
\vspace{-0.25cm}

\textbf{Machine Learning:} Experienced with ML and LLMs libraries (\texttt{transformers},\texttt{spaCy},\texttt{PyTorch},\texttt{sklearn}, and others).
\vspace{-0.75cm}

\textbf{Sysadmin:} Skilled \texttt{Linux} enthusiast experienced in remote computing resources (\texttt{SSH/SFTP}, \texttt{Bash}, \texttt{Slurm}).
\end{rSection}

\vspace{-0.3cm}
\begin{rSection}{Education}

{\bf Ph.D. in Computer Science}, GPA 3.94/4, Universidad Nacional del Sur, Argentina \hfill {2017 --- 2021}\\
\underline{Thesis}: Variable selection and causal discovery from  newspaper article texts.\\
% Relevant courses: Econometrics, 

\vspace{-0.6cm}


{\bf B.Sc in Computer Systems Engineering}, GPA 3.51/4, Universidad Nacional del Sur, Argentina \hfill {2011 --- 2016}\\
% \underline{Final project}: Feature engineering techniques evaluated on supervised learning methods.
%Minor in Linguistics \smallskip \\
%Member of Eta Kappa Nu \\
%Member of Upsilon Pi Epsilon \\
\vspace{-0.2cm}


\end{rSection}

\vspace{-0.4cm}
\begin{rSection}{EXPERIENCE}



% CAPE BRETON
\textbf{Postdoc Fellow  } \hfill Aug 2023 --- Aug 2024\\\underline{Cape Breton University} \hfill \textit{Sydney, NS}\\
\vspace{-0.7cm}

\begin{itemize}
    \item \textbf{Research Focus}: Developing explainable ML models to analyze real-world healthcare data from Nova Scotia.
    \vspace{-0.25cm}
    \item \textbf{Objectives}: Identify risk factors and create guidelines to tackle Unplanned Hospital Readmissions (UHR).
    \vspace{-0.25cm} 
    \item \textbf{Responsibilities}: Handle the entire research process, including formulating research questions, conducting literature reviews, executing experiments, writing scientific papers, and presenting findings in relevant venues. 
    \vspace{-0.25cm}
    \item \textbf{Achievements}: Built a guideline that reduced the required screening in half while capturing 72\%+ UHR.
\end{itemize}
% Our research focused on using explainable ML applied to the real-world data in the healthcare domain to identify risk factors and obtain insight for medical personnel to handle the unplanned hospital readmission problem in Nova Scotia. I handle the entire lifecycle of our research. From formulating research questions, conducting thorough literature reviews, executing experiments to produce both theoretically grounded and applicable solutions, and writing scientific articles and presenting our results in relevant venues.

% \begin{itemize}
%     \item Research focused on explainable ML applied to real-world healthcare data from Nova Scotia, Canada.
%     \vspace{-0.35cm}
%     \hspace{-1cm} \item  Aim: Identify risk factors and insights for medical personnel to handle unplanned hospital readmissions.
%     \vspace{-0.35cm}
%     \item Responsibilities:
%     \vspace{-0.35cm}
%     \begin{itemize}
%             \item Formulating research questions.
%             \vspace{-0.35cm}
%             \item Conducting thorough literature reviews.
%             \vspace{-0.35cm}
%             \item Executing experiments to produce theoretically grounded and applicable solutions.
%             \vspace{-0.35cm}
%             \item Writing scientific articles and presenting results in relevant venues.
%     \end{itemize}
% \end{itemize}


% \begin{itemize}
%     \item \textit{\textcolor{darkgray}{Mariano Maisonnave, Enayat Rajabi, Majid, and Peter, 2024. Alternate Level of Care Patients in Canada: A Systematic Review: Alternate Level of Care Patients in Canada. Accepted for publication in Canadian Geriatrics Journal.}}
% \end{itemize}
% \vspace{-0.2cm}
\textbf{Postdoc Fellow  } \hfill Jan 2022 --- July 2023\\
\underline{Dalhousie University} \hfill \textit{Halifax, NS}\\

\vspace{-0.7cm}
\begin{itemize}
    \item \textbf{Research Focus}: To review the state-of-the-art in active learning for high-recall information and to provide solution to social science researchers conducting their research using large amounts of textual data.
    \vspace{-0.25cm}
    \item \textbf{Responsibilities}: Designed and implemented an active learning system in Python that enabled researchers to interact with a ML model and train it in real-time. 
    \vspace{-0.25cm}
    \item \textbf{Achievements}: Social science researchers could perform information retrieval of over 2.9+ million news articles using our system, which led them to conduct novel research in their fields. Results were organized through LLMs.
\end{itemize}

% I reviewed the state-of-the-art active learning for high-recall information retrieval to provide a solution to social science researchers conducting their research using large amounts of textual data. I designed and implemented an active learning system in Python that enabled researchers to interact with a ML model and train it in real-time. Social science researchers could perform information retrieval over millions of news articles using our system, which led them to conduct novel research in their fields.
% \begin{itemize}
%     \item Reviewed state-of-the-art active learning for high-recall information retrieval.
    
%     \item Aim: Provide a solution for social science researchers using large amounts of textual data.
%     \item Responsibilities:
%     \item Designed and implemented an active learning system in Python.
%     \item Enabled researchers to interact with a ML model and train it in real-time.
%     \item System allowed information retrieval over millions of news articles.
%     \item Facilitated novel research in social science fields.
% \end{itemize}


\textbf{Research Fellow} \hfill Apr 2017 --- Oct 2021\\
\underline{Institute for Computer Science and Engineering} \hfill \textit{Bahía Blanca, Argentina}\\
\vspace{-0.7cm}
\begin{itemize}
    \item \textbf{Overview}: Interdisciplinary project that aims to extract causal graphs from a large corpus of news article texts using causality learning and time series analysis tools (primarily in Python, some in R), NLP, and ML tools.
    \vspace{-0.25cm}
    \item \textbf{Achievements: }
    \vspace{-0.25cm}
    \begin{itemize}
    \item  Built and evaluated an ML system that leveraged LLMs to identify ongoing events and relevant variables from news articles, obtaining a 13.3\% improvement in F1-score over existing event detection techniques.
    \vspace{-0.25cm}
    \item Condensed information from large text to single causal graph using causal learning tools with concepts/events.
    \vspace{-0.75cm}
    \item Applied causal learning tools to a real-world energy demand time-series dataset to understand the role of causality in forecasting analysis, reducing the forecast error by 12 to 27\% using causality-oriented methods.
    \vspace{-0.7cm}
    \item Published articles on TW, causal learning, NLP, and energy forecasting in high-impact journals/conferences.
    \end{itemize}
\end{itemize}

% I collaborated closely with Computer Science and Economics researchers to design a framework to extract causal graphs from a large corpus of news article texts. I developed a term-weighting technique to identify relevant concepts present in texts. I built and evaluated a ML model to identify ongoing events from news articles. We used causal learning tools on top of the extracted variables (concepts and events) to condense the information from the text into a single causal graph. As a case study, we also applied causal learning tools to a real-world energy demand time-series dataset to understand the role of causality in forecasting analysis. 

% I used causality learning and time series analysis tools (primarily in Python, some in R), natural language processing, and ML tools to conduct the project. 
% We published articles on term weighting, causal learning, natural language processing, and energy forecasting in several high-impact journals and conferences.





\textbf{Teaching} \hfill Oct 2015 --- Dec 2021\\
\underline{Universidad Nacional del Sur} \hfill \textit{Bahía Blanca, Argentina}\\
\vspace{-0.7cm}

\begin{itemize}
    \item Teaching assistant, head of teaching assistant, and professor at several university undergraduate courses.
    \vspace{-0.25cm}
    \item \textbf{Courses}: Object-oriented Programming, Computer Networking, and Embedded Systems. 
\end{itemize}
%  \begin{itemize}
%     \itemsep -3pt {} 
%      \item Work as teaching assistant, head of teaching assistant and professor of several undergrad courses.
%      \item Courses: computer networking, embedded systems and introduction to object-oriented programming. 

%  \end{itemize}
\end{rSection} 
%----------------------------------------------------------------------------------------
%	EDUCATION SECTION
%----------------------------------------------------------------------------------------




%----------------------------------------------------------------------------------------
% TECHINICAL STRENGTHS	
%----------------------------------------------------------------------------------------




% \begin{rSection}{SELECTED PUBLICATIONS}
% \vspace{-1.25em}
% \item \textbf{Maisonnave, M}., Delbianco, F., Tohmé, F., Milios, E. and Maguitman, A.G., 2022. \textit{Causal graph extraction from news: a comparative study of time-series causality learning techniques}. PeerJ Computer Science, 8, p.e1066. \href{https://peerj.com/articles/cs-1066/}{[link]}
% \vspace{-0.2cm}
% \item \textbf{Maisonnave, M}., Delbianco, F., Tohmé, F., Maguitman, A. and Milios, E., 2022. \textit{Detecting ongoing events using contextual word and sentence embeddings}. Expert Systems with Applications, 209, p.118257. \href{https://www.sciencedirect.com/science/article/pii/S0957417422013975}{[link]}
% \vspace{-0.2cm}
% \item \textbf{Maisonnave, M}., Delbianco, F., Tohmé, F. and Maguitman, A., 2021. \textit{Assessing the behavior and performance of a supervised term-weighting technique for topic-based retrieval}. Information Processing \& Management, 58(3), p.102483. \href{https://www.sciencedirect.com/science/article/pii/S0306457320309729}{[link]}
% % \vspace{-0.15cm}
% \vspace{-0.2cm}
% \end{rSection} 
%----------------------------------------------------------------------------------------
%	WORK EXPERIENCE SECTION
%----------------------------------------------------------------------------------------

% \begin{rSection}{PROJECTS}
% \vspace{-1.25em}
% \item \textbf{Hiring Search Tool.} {Built a tool to search for Hiring Managers and Recruiters by using ReactJS, NodeJS, Firebase and boolean queries. Over 25000 people have used it so far, with 5000+ queries being saved and shared, and search results even better than LinkedIn! \href{https://hiring-search.careerflow.ai/}{(Try it here)}}
% \item \textbf{Short Project Title.} {Build a project that does something and had quantified success using A, B, and C. This project's description spans two lines and also won an award.}
% \item \textbf{Short Project Title.} {Build a project that does something and had quantified success using A, B, and C. This project's description spans two lines and also won an award.}
% \end{rSection} 

%----------------------------------------------------------------------------------------
% \begin{rSection}{Extra-Curricular Activities} 
% \begin{itemize}
%     \item 	Actively write \href{https://www.faangpath.com/blog/}{blog posts} and social media posts (\href{https://www.tiktok.com/@faangpath}{TikTok}, \href{https://www.instagram.com/faangpath/?hl=en}{Instagram}) viewed by over 20K+ job seekers per week to help people with best practices to land their dream jobs. 
%     \item	Sample bullet point.
% \end{itemize}


% \end{rSection}

%----------------------------------------------------------------------------------------


% \newpage
% \begin{rSection}{PUBLICATIONS} 
% \begin{itemize}    
%     \item Maisonnave, M., Delbianco, F., Tohmé, F., Milios, E. and Maguitman, A., 2024. Learning causality structures from electricity demand data. Energy Systems, pp.1-23.
%     \vspace{-0.15cm}
%     \item Maisonnave, M., Delbianco, F., Tohme, F., Milios, E. and Maguitman, A.G., 2022. Causal graph extraction from news: a comparative study of time-series causality learning techniques. PeerJ Computer Science, 8, p.e1066.
%     \vspace{-0.15cm}
%     \item Ramirez-Orta, J.A., Sabando, M.V., Maisonnave, M. and Milios, E.E., 2022. MALNIS at IberLEF-2022 DETESTS Task: A Multi-Task Learning Approach for Low-Resource Detection of Racial Stereotypes in Spanish. In IberLEF@ SEPLN.
%     \vspace{-0.15cm}
%     \item Maisonnave, M., Delbianco, F., Tohmé, F., Maguitman, A. and Milios, E., 2022. Detecting ongoing events using contextual word and sentence embeddings. Expert Systems with Applications, 209, p.118257.
%     \vspace{-0.15cm}
%     \item Maisonnave, M., Delbianco, F., Tohmé, F. and Maguitman, A., 2021. Assessing the behavior and performance of a supervised term-weighting technique for topic-based retrieval. Information Processing \& Management, 58(3), p.102483.
%     \vspace{-0.15cm}
%     \item Maisonnave, M., Delbianco, F., Tohmé, F., Maguitman, A.G. and Milios, E.E., 2020, September. Assessing causality structures learned from digital text media. In Proceedings of the ACM Symposium on Document Engineering 2020 (pp. 1-4).
%     \vspace{-0.15cm}
%     \item Maisonnave, M., Delbianco, F., Tohmé, F.A. and Maguitman, A.G., 2019. A flexible supervised term-weighting technique and its application to variable extraction and information retrieval. Inteligencia Artificial, 22(63), pp.61-80.

%     % \item Recipient of a five-year scholarship to pursue a doctorate in computer science, 2017-2022. Granted by the Argentinian National Scientific and Technical Research Council (CONICET).
%     % \vspace{-0.2cm}     
%     % \item Recipient of the Encouragement of Scientific Vocations Scholarship (EVC-CIN), 2015 and 2016. Funding for undergrad students to conduct a part-time research project.
% \end{itemize}
% \end{rSection}






\vspace{-0.2cm}
\begin{rSection}{AWARDS AND SCHOLARSHIPS} 
\begin{itemize}
    \item Google Latin America Research Award (LARA) Winner: Awarded three consecutive years (2019, 2020, 2021).
    \vspace{-0.25cm}
    \item Emerging Leaders in the Americas Program (ELAP) Scholarship Recipient, 2018.
    % \item Recipient of a five-year scholarship to pursue a doctorate in computer science, 2017-2022. Granted by the Argentinian National Scientific and Technical Research Council (CONICET).
    % \vspace{-0.2cm}     
    % \item Recipient of the Encouragement of Scientific Vocations Scholarship (EVC-CIN), 2015 and 2016. Funding for undergrad students to conduct a part-time research project.
\end{itemize}
\end{rSection}


% %----------------------------------------------------------------------------------------
% \begin{rSection}{Leadership} 
% \begin{itemize}
%     \item Admin for the \href{https://discord.com/invite/WWbjEaZ}{FAANGPath Discord community} with over 6000+ job seekers and industry mentors. Actively involved in facilitating online events, career conversations, and more alongside other admins and a team of volunteer moderators! 
% \end{itemize}


% \end{rSection}


\end{document}
